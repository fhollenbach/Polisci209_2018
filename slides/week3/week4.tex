% Created 2018-09-24 Mon 08:30
% Intended LaTeX compiler: pdflatex
\documentclass[presentation]{beamer}
\usepackage[utf8]{inputenc}
\usepackage[T1]{fontenc}
\usepackage{graphicx}
\usepackage{grffile}
\usepackage{longtable}
\usepackage{wrapfig}
\usepackage{rotating}
\usepackage[normalem]{ulem}
\usepackage{amsmath}
\usepackage{textcomp}
\usepackage{amssymb}
\usepackage{capt-of}
\usepackage{natbib}
\usepackage[linktocpage,pdfstartview=FitH,colorlinks,
linkcolor=blue,anchorcolor=blue,
citecolor=blue,filecolor=blue,menucolor=blue,urlcolor=blue]{hyperref}
\setbeamertemplate{frame footer}{\insertshortauthor}
\setbeamerfont{page number in head/foot}{size=\tiny}
\setbeamercolor{footline}{fg=gray}
\author{Florian Hollenbach}
\usepackage[english]{isodate}
\usepackage{amsmath,amsthm,amssymb,amsfonts}
\usetheme{metropolis}
\usecolortheme{}
\usefonttheme{}
\useinnertheme{}
\useoutertheme{}
\author{Florian Hollenbach}
\date{\today}
\title{Political Science 209 - Fall 2018}
\subtitle{Observational Studies}

\hypersetup{
 pdfauthor={Florian Hollenbach},
 pdftitle={Political Science 209 - Fall 2018},
 pdfkeywords={},
 pdfsubject={},
 pdfcreator={Emacs 25.3.1 (Org mode 9.1.9)}, 
 pdflang={English}}
\begin{document}

\maketitle


\begin{frame}[label={sec:org05b4cd5}]{Review}
\begin{block}{What is the fundamental problem of causal inference?}
\end{block}
\end{frame}

\begin{frame}[label={sec:org44116bf}]{Review}
\begin{block}{What about randomized control trials allows us to credibly estimate a causal effect?}
\end{block}
\end{frame}

\begin{frame}[label={sec:orgbcc2da1}]{Get out the Vote Study}
\begin{block}{What can induce citizens to vote?}
\end{block}
\end{frame}

\begin{frame}[label={sec:org9897a0c}]{What was the experiment?}
\pause
Letters to randomized households with treatment:
\begin{enumerate}
\item Naming and Shaming: your neighbors will know
\item Civic Duty
\item Hawthorne Effect Message
\item Control (no letter)
\end{enumerate}
\end{frame}

\begin{frame}[label={sec:org842d313}]{Let's go to R-studio quick}
\end{frame}


\begin{frame}[label={sec:org1429165}]{Observational Studies and Causal Inference}
\begin{block}{What is the main problem for observational studies?}
\pause
\begin{itemize}
\item Confounders: variables that are associated with both treatment and outcome
\end{itemize}
\end{block}
\end{frame}

\begin{frame}[label={sec:org824234e}]{What is the Problem with Confounders?}
\pause

\begin{itemize}
\item If pre-treatment characteristics are associated with treatment and outcome, we can't disentangle causal effect from confounding bias
\end{itemize}

\pause

\begin{itemize}
\item Selection into treament example: Maybe minimum wage was increased because unemployment was particularly low in NJ, but not PA
\end{itemize}
\end{frame}

\begin{frame}[label={sec:org0bdd866}]{Examples of Confounding}
\begin{itemize}
\item Are incumbents more likely to win elections? Yes, but\ldots{}
\end{itemize}

\pause

\begin{itemize}
\item Incumbents receive more campaign contributions
\item Incumbents have more staff
\end{itemize}
\end{frame}

\begin{frame}[label={sec:orga5a8ffb}]{Examples of Confounding}
\begin{itemize}
\item Does higher income lead countries to democratize?
\end{itemize}

\pause

\begin{itemize}
\item Higher income countries have more educated populations
\end{itemize}
\end{frame}



\begin{frame}[label={sec:org8f55328}]{What can we do about confounding in observational studies?}
\pause

\begin{itemize}
\item Make \emph{Treatment} and \emph{Control} groups as similar to each other as possible
\item Especially on variables that might matter for treatment status and outcome
\item Analyze subsets or \emph{statistical control}, such that we compare treated and control units that have same value on confounder
\end{itemize}
\end{frame}

\begin{frame}[label={sec:orga75961c}]{Another problem with observational studies:}
\begin{itemize}
\item Reverse causality
\end{itemize}

\pause

\begin{itemize}
\item Example: Does economic growth cause democratization or democratization cause growth?
\end{itemize}

Why do experiments not suffer from the threat of reverse causality?
\end{frame}

\begin{frame}[label={sec:org58188ea}]{Observational studies}
Difference-in-Differences Design
\end{frame}

\begin{frame}[label={sec:org7b310af}]{Difference-in-Differences Design}
\begin{itemize}
\item Compare trends before and after the treatment across the same units
\item Takes initial conditions into account
\end{itemize}
\end{frame}

\begin{frame}[label={sec:org0529ff8}]{Difference-in-Differences Design}
\begin{itemize}
\item Need data measured for both treatment and control at two different time periods: before and after treatment
\end{itemize}
\begin{center}
\includegraphics[width=4cm]{/Users/florianhollenbach/Documents/GitHub/Polisci209_2018/slides/week3/Illustration_of_Difference_in_Differences.png}
\end{center}

\begin{itemize}
\item Total difference between P2 and S2 can not be attributed to treatment. Why?
\end{itemize}
\end{frame}

\begin{frame}[label={sec:org8831d9b}]{Difference-in-Differences Design}
\begin{center}
\includegraphics[width=4cm]{/Users/florianhollenbach/Documents/GitHub/Polisci209_2018/slides/week3/Illustration_of_Difference_in_Differences.png}
\end{center}

What might be a necessary condition for Diff-in-Diff to work?

\pause

Parralel Trends Assumptions
\end{frame}

\begin{frame}[label={sec:orgb577522}]{Difference-in-Differences Design}
\begin{center}
\includegraphics[width=8cm]{/Users/florianhollenbach/Documents/GitHub/Polisci209_2018/slides/week3/did.png}
\end{center}
\end{frame}


\begin{frame}[label={sec:org2bfeff3}]{Describing numeric variables:}
\begin{itemize}
\item Mean
\item Median
\item Quantiles
\end{itemize}
\end{frame}


\begin{frame}[label={sec:org930485f}]{Quantiles}
\begin{itemize}
\item splitting observations into equaly size groups, e.g., quartiles, quantiles
\item 75th percentile is the threshold under which 75\% of observations lie
\item What percentile is the median?
\end{itemize}
\end{frame}


\begin{frame}[label={sec:org2c6e110}]{Describing the spread of numeric variables:}
\begin{itemize}
\item IQR:
\end{itemize}

\pause

Difference between 75th percentile and 25th percentile
\end{frame}

\begin{frame}[label={sec:org98486db}]{Describing the spread of numeric variables:}
Standard Deviation

\pause


SD = \(\sqrt{\frac{1}{n} \sum^{N}_{i = 1} (x_{i} - \bar{x})^{2}}\)
\end{frame}

\begin{frame}[label={sec:org882fcaa}]{Standard Deviation}
\begin{center}
\includegraphics[width=8cm]{/Users/florianhollenbach/Documents/GitHub/Polisci209_2018/slides/week3/sd.png}
\end{center}
\end{frame}


\begin{frame}[label={sec:org3cb0f30}]{Describing single Variables}
\begin{itemize}
\item Barplots can be used to summarize factor(?) variables
\item Proportion of observations in each category as the height of each bar
\end{itemize}
\end{frame}

\begin{frame}[label={sec:org5f04d8d}]{Barplots}
\begin{center}
\includegraphics[width=8cm]{/Users/florianhollenbach/Documents/GitHub/Polisci209_2018/slides/week3/barplot.png}
\end{center}
\end{frame}


\begin{frame}[label={sec:org5cddf26}]{Histograms}
\begin{itemize}
\item Histograms look similar to barplots
\item Used for numeric variables
\item Numeric variables are \emph{binned} into groups
\end{itemize}
\end{frame}

\begin{frame}[label={sec:org3406ad8}]{Histograms}
\begin{itemize}
\item Each bar is for one bin
\item Height of each bar is the \emph{density} of the bin
\end{itemize}

\pause

\begin{itemize}
\item Important: Height is share of observations in bin divided by bin size
\end{itemize}

\pause

\begin{itemize}
\item Unit of vertical axis (y-axis) is interpreted as percentage per horizontal (x-axis) unit
\end{itemize}
\end{frame}

\begin{frame}[label={sec:org9481963}]{Histograms}
\begin{itemize}
\item Area of each bar is the share of observations that fall into that bin
\item Area of all bins sum to one
\end{itemize}
\end{frame}

\begin{frame}[label={sec:org6019939}]{Histograms}
\begin{center}
\includegraphics[width=8cm]{/Users/florianhollenbach/Documents/GitHub/Polisci209_2018/slides/week3/histogram.pdf}
\end{center}
\end{frame}


\begin{frame}[label={sec:org638494d}]{Boxplots}
\begin{itemize}
\item Boxplots also display the distribution of a numeric variable
\item Boxplots show the \emph{median}, \emph{quartiles}, and \emph{IQR}
\end{itemize}
\end{frame}

\begin{frame}[label={sec:orgf73f02c}]{Boxplots}
\begin{center}
\includegraphics[width=8cm]{/Users/florianhollenbach/Documents/GitHub/Polisci209_2018/slides/week3/boxplot.png}
\end{center}
\end{frame}


\begin{frame}[label={sec:orge9cc12b}]{Boxplots can show how two variables covary}
\begin{center}
\includegraphics[width=8cm]{/Users/florianhollenbach/Documents/GitHub/Polisci209_2018/slides/week3/boxplot2.pdf}
\end{center}
\end{frame}

\begin{frame}[label={sec:orgb6a874d}]{Survey Sampling}
\begin{itemize}
\item A sample is a small share of the population in that we are interested in
\end{itemize}

\pause

\begin{itemize}
\item How do we draw samples in such a way that polls accurately reflect what is going to happen?

\item How to construct samples that will represent the population?
\end{itemize}
\end{frame}

\begin{frame}[label={sec:org87e90ac}]{Survey Sampling}
\begin{itemize}
\item Example: We want to know the voting intentions of Texans (or Americans)
\item We can hardly ask all eligible voters about their intention
\end{itemize}

\pause
\begin{itemize}
\item We take a \emph{sample}
\end{itemize}
\end{frame}

\begin{frame}[label={sec:org9a96577}]{Survey Sampling}
\begin{itemize}
\item The size of the sample is less important than its composition
\end{itemize}

\begin{center}
\includegraphics[width=8cm]{/Users/florianhollenbach/Documents/GitHub/Polisci209_2018/slides/week3/drudge.jpeg}
\end{center}

\begin{itemize}
\item Never trust an internet poll
\end{itemize}
\end{frame}

\begin{frame}[label={sec:org2e6b4fe}]{Literary Digest Sample}
\begin{itemize}
\item Mail questionnaire to 10 million people

\item Addresses came from phone books and club memberships

\item Problems?
\end{itemize}

\pause

\begin{itemize}
\item Biased \emph{sample}
\end{itemize}
\end{frame}


\begin{frame}[label={sec:org47e3b43}]{Quota Samping}
\begin{itemize}
\item Sample certain groups until quota is filled

\item Does not mean unobservables are representative
\end{itemize}
\end{frame}

\begin{frame}[label={sec:org27b2827}]{Simple Random Sampling}
\begin{itemize}
\item Think of all voters sitting in a box, survey firm randomly draws voters

\item Random draws without replacement give us an unbiased estimate of the population

\item Everybody has the same chance of being in the sample
\end{itemize}
\end{frame}


\begin{frame}[label={sec:org1ede10f}]{Simple Random Sampling}
\begin{itemize}
\item Pre-determined number of units are randomly selected from population

\item Sample will be representative of population on observed and unobserved characteristics
\end{itemize}
\end{frame}



\begin{frame}[label={sec:orgf9ede7f}]{Simple Random Sampling}
\begin{itemize}
\item Not every single sample will be exactly representative

\item If we were to take a lot of random samples (say 1000 samples of 1000 respondents), on average the samples would be representative
\end{itemize}
\end{frame}

\begin{frame}[label={sec:org555ea27}]{Simple Random Sampling}
\begin{itemize}
\item Each single sample can be off and different

\item Polls are associated with uncertainty
\end{itemize}

\begin{center}
\includegraphics[width=8cm]{/Users/florianhollenbach/Documents/GitHub/Polisci209_2018/slides/week3/cruz.png}
\end{center}

\pause


\begin{center}
\includegraphics[width=8cm]{/Users/florianhollenbach/Documents/GitHub/Polisci209_2018/slides/week3/orourke.png}
\end{center}
\end{frame}


\begin{frame}[label={sec:orgb740948}]{Random Sampling is hard}
\begin{itemize}
\item How to create sampling frame?

\item Random digit dialing? Walking to random houses?

\item Multi-stage cluster sampling
\end{itemize}
\end{frame}

\begin{frame}[label={sec:org14f39ff}]{Non-reponse bias}
\begin{itemize}
\item Unit non-response bias:
\end{itemize}

\begin{center}
\includegraphics[width=8cm]{/Users/florianhollenbach/Documents/GitHub/Polisci209_2018/slides/week3/cohn.png}
\end{center}
\end{frame}




\begin{frame}[label={sec:org40a7991}]{Non-reponse bias}
\begin{itemize}
\item Item non-response bias:
\emph{What was the last crime you committed?}
\item Sensitive questions: non-response, social desirability bias
\emph{Turnout}, \emph{racial prejudice}, \emph{corruption}
\end{itemize}
\end{frame}


\begin{frame}[label={sec:org8c49026}]{Why could this be a problem in the Afghanistan example?}
\begin{center}
\includegraphics[width=8cm]{/Users/florianhollenbach/Documents/GitHub/Polisci209_2018/slides/week3/afghan-survey.jpg}
\end{center}
\end{frame}
\end{document}